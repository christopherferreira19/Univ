Après l'analyse sémantique et avant l'éxecution, le programme est compilé dans un
langage intermédiaire semblable à un langage d'assemblage ou à un bytecode. Ce
langage intermédiaire est ensuite interpreté par une ``machine virtuelle''
fonctionnant sur une base de pile. (Plus exactement deux piles mais on aurait très
bien pu se ramener à une unique pile).

En effet, on peut grossièrement assimiler le langage intermédiaire à une transformation
en notation postfixée du programme source facilement interpretable par une machine à pile.
Par exemple l'expression :
\begin{verbatim}
x = 5 * 6 + 3
\end{verbatim}
transformée en un arbre syntaxique abstrait :
\begin{verbatim}
Affectation
   x
   Somme
      Produit
         5
         6
      3
\end{verbatim}
peut ensuite être compilée sous la forme
\begin{verbatim}
empiler 5
empiler 6
faire Produit
empiler 3
faire Somme
depiler/stocker dans x
\end{verbatim}
en assumant que chaque opération (et par extension chaque fonction) prend
ses arguments en sommet de pile et y met à la place sa valeur de retour.

\hfill\break
Le jeu d'instruction du langage intermédiaire/de la machine virtuelle est défini
sur ce principe :
\begin{description}
   \item[EPL <valeur littéral>:] Met la <valeur littéral> au sommet de la pile d'appel
   \item[DPL:] Dépile le sommet de la pile d'appel
   \item[CHR <indice variable>:] Charge la variable à <indice variable> et sur la pile d'appel
   \item[STK <indice variable>:] Stocker le sommet de la pile (en dépilant) à <indice variable>
   \item[PRM <nom primitive>:] Met primitive <nom primitive> au sommet de la pile d'appel
   \item[CLT <nom de la routine> {[$\sim$]}<nombre de variables à cloturer>:] Définition d'une fonction
   mise au sommet de la pile d'appel
   \item[STI <decalage>:] Saut inconditionnel
   \item[STV <decalage>:] Saut si vrai dépilé de la pile d'appel
   \item[STF <decalage>:] Saut si faux dépilé de la pile d'appel
   \item[APP:] Appel de la fonction sur la pile
   \item[RET:] Fin de la fonction en cours et retour à la fonction appelante
\end{description}

Pour representer le code des fonctions sous la forme de ce langage intermédiaire on
utilise le concept de routine qui est la suite statique d'instructions d'une fonction.
Les routines sont définies par un bloc commencant par le nom entre crochet suivi
d'un nombre representant l'allocation nécessaire à la fonction.

Une fonction est ainsi la donnée dynamique constituée d'une réference vers une
routine et d'un ensemble de variables enfermées. Elles sont créées par l'instruction
\textbf{CLT} qui associe ces deux composantes. Le caractère `$\sim$' devant le
nombre de variables à capturer indique que la fonction est recursive.

Les variables locales de chaque fonction sont representées par la deuxième pile
fonctionnant par frame. Autrement dit, chaque fonction défini un nombre de variables
locales dont elle a besoin (une allocation) et au moment de l'appel de la fonction
la machine virtuelle déplace le curseur de la pile de la taille demandée et la fonction
peut referencer ses variables en fonction d'indices determinés statiquement vers cette frame.

L'argument des instructions de saut est défini en terme de nombre d'instructions de décalage
par rapport à l'instruction de saut en elle-même. Elles sont suffisantes pour representer
les trois structures de contrôle et les deux opérateurs paresseux `et' et `ou' (uniquement STV
aurait même suffit).

Le langage intermédiaire contient aussi des annotations informatives pour le nom des
variables et les positions correspondantes du code source.

\paragraph{Example de compilation}
\begin{verbatim}
   [] 4
   EPL 1 @1:14
   STK 0 &fact_borne @1:12
   EPL 1 @2:15
   STK 1 &fact_defaut @2:13
   EPL 5 @3:13
   STK 2 &fact_test @3:11
   CHR 0 &fact_borne @5:8
   CHR 1 &fact_defaut @5:8
   CLT fact0 ~2 @5:8
   STK 3 &fact @5:6
   CHR 2 &fact_test @10:16
   CHR 3 &fact @10:11
   APP @10:15
   PRM ecrire_ln @10:10
   APP @10:10
   EPL unit @1:1
   RET @1:1

   [fact0] 4
   STK 3 &fact0 @5:8
   STK 2 &fact_defaut @5:8
   STK 1 &fact_borne @5:8
   STK 0 &n @5:8
   CHR 0 &n @6:7
   CHR 1 &fact_borne @6:12
   PRM inferieur_egale @6:9
   APP @6:9
   STF 4 @6:4
   CHR 2 &fact_defaut @6:32
   RET @6:25
   STI 1 @6:4
   CHR 0 &n @7:11
   CHR 0 &n @7:20
   EPL 1 @7:24
   PRM moins @7:22
   APP @7:22
   CHR 3 &fact @7:15
   APP @7:19
   PRM multiplier @7:13
   APP @7:13
   RET @7:4
\end{verbatim}
