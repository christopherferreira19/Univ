\subsection{Affectations}

La gestion des affectations par le langage est extrêmement laxiste,
il est possible d'affecter un entier à une variable puis de lui
affecter une chaine de caractères plus loin voire de lui assigner deux
valeurs de types différents dans deux branches alternatives de code.
Néanmoins l'analyseur sémantique garde trace du type de chaque
variable au fur et à mesure du flot du programme et effectue une
vérification précise de la compatibilité entre les types des variables
et leurs utilisations.

\subsection{Fonction de première classe et fermeture}

Le langage supporte la définition de nouvelles fonctions. De plus,
ces fonctions sont de première classe (manipulables comme des
valeurs au même titre qu'un entier ou une chaîne de caractères)
et supporte le principe de fermeture (par portée lexicale). Elles
sont définies exclusivement anonymement (ce n'est qu'en
étant assignée à une variable qu'elles obtiennent un nom).
La fermeture des variables au moment de la définition de la fonction
est réalisée par `copie'. Autrement dit, les nouvelles affectations
a posteriori de la variable enfermée n'affectent pas la fonction :
\begin{verbatim}
x = 1
f = fn -> Entier {
   retour x
}

x = 2
assert(f() == 1)
\end{verbatim}

\subsection{Système de Type}
Les types sont définis récursivement à partir des types de base.
Les types de base sont les types concrets présents à l'exécution :
\begin{description}
   \item[Unit:] Type singleton de la valeur unit
   \item[Vrai:] Type singleton de la valeur vrai
   \item[Faux:] Type singleton de la valeur faux
   \item[Entier:] Type des nombres entiers
   \item[Flottant:] Type des nombres réels representés avec virgule flottante
   \item[Chaine:] Type des chaînes de caractères
\end{description}

\hfill\break
À partir de ces types peut être construit de nouveaux types
avec les deux constructions suivantes :
\begin{description}
   \item[Union:] Etant donné deux types A et B leur union (notée `A|B')
      représente un nouveau type. Dans la correspondance de Curry-Howard,
      il represente le `ou' logique; ie. avec P(T) le prédicat associé à
      un type T, on a ``P(A|B) $\iff$ P(A) $\cup$ P(B)''.
   \item[Fonction:] Ce type est nécessaire pour representer les fonctions
      qui sont de première classe. Il est notée Fn(Type\_Arg1, ...) -> Type\_Retour
      et encode le nombre d'arguments, leur types et celui du retour de la fonction.
\end{description}

\hfill\break
Pour raisonner quant au types des variables le langage utilise la 
notion de généralisation. Un type A généralise un autre type B si
toute valeur assignable au type B est assignable au type A.
On remarque :
\begin{itemize}
   \item[$\bullet$] $\forall$A, A généralise A
   \item[$\bullet$] $\forall$A,B, A|B généralise A et B
   \item[$\bullet$] Fn(A1, ..., An) -> R généralise Fn(B1, ..., Bn) -> S
      si et seulement si, $\forall$i, Bi généralise Ai et R généralise S.
      (Notons l'inversion pour le type des arguments. Si l'on assimile le
      type fonction à un type paramètrique, on retrouve le problème de la variance,
      i.e: Une fonction est covariante en son type de retour et contravariante en 
      les types de ses arguments)
\end{itemize}

\hfill\break
A partir de ce principe de généralisation est défini un nouveau type, le type Tous
qui est le type qui généralise tout les autres types. (Ou de manière équivalente, 
le type qui est l'union de tous les autres types).

\hfill\break
Les déclarations de types ne sont nécessaires que dans les déclarations de fonctions.
Tout les autres types peuvent être inférés à partir des valeurs littérales et des
types des fonctions.

\hfill\break
\textbf{Problème des fonctions récursives :}
Tel qu'est défini le langage les fonctions récursives posent problèmes. En effet,
l'analyse sémantique est purement sequentielle et les fonctions n'ont un nom qu'après
avoir été analysée (si elles sont assignées à une variable). Pour supporter les fonctions
récursive l'analyseur est obligé de considérer le fait qu'une fonction est sur le
point d'être affectée à une variable et de faire passer le nom de la variable en question
à l'analyse de la fonction pour que cette dernière puissent reconnaitre les occurences
d'appels recursifs. Une fonction peut ainsi être flaggée comme récursive et être
considérée comme s'enfermant elle-même.
Il faut noter que cette solution ne fonctionne pas pour les fonctions mutuellement
récursives, par exemple :
\begin{verbatim}
pair = fn(n: Entier) -> Booleen {
   si n == 0 { retour vrai }
   retour impair(n - 1)
}

impair = fn(n: Entier) -> Booleen {
   si n == 0 { retour faux }
   retour pair(n - 1)
}
\end{verbatim}
La première occurence de la variable `impair' induit une erreur car elle est considérée
non définie par le compilateur.
